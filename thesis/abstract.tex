\chapter{Abstract}

Balancing the workload of an assembly line
leads to a decision problem where the aim is to
assign a set of tasks to a sequence of work stations,
whilst optimising a measure of performance, \eg maximising the production
rate.
The SetUp Assembly Line Balancing and Scheduling Problem (\sua{})
is a recent variant on the classical Assembly Line Balancing Problem (\albp{}).
This new problem introduces sequence-dependent
setup times between any pair of consecutive tasks
processed within a work station's task sequence.
Considering these setup times leads an additional layer
of complexity as the problem now involves a set
of scheduling decisions for each station along the production line.

Benders decomposition naturally suggests itself as an appropriate
approach for the \sua{} as we can divide the problem's decisions
into two distinct sets.
We propose a general logic-based Benders decomposition framework which 
leads to the assignment portion being decided by a relaxed
master problem and a number of sub-problems which together
handle the scheduling portion.
Each sub-problem is highly combinatorial in nature and so
we consider state-of-the-art solving technology from
both Operations Research and Artificial Intelligence
to solve the scheduling problem efficiently.
% We provide a detailed experimental analysis into the superior 
% sub-problem formulation
A range of Benders cuts are devised 
and their practical effectiveness
is one of our primary contributions to the 
literature.

We test the best-known exact methods to solve
the \sua{} and use these models
as a benchmark against which to scrutinise our hybrid solution methodology.
In total, 396 instances of the problem were tested and our Benders
decomposition approach vastly outperformed the best models from the literature.
Due to this variant of the \albp{} being notably under-studied,
we conclude the paper by remarking on a range of possible future directions
which the research community could investigate.
